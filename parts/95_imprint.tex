% BOF

\zchinrsetup{
  reset,
  article = {
    title style  = {category},
    header style = {category},
    category     = {Impressum},
    pagination   = {none}
  }
}

\printtitle

\begin{imprint}
  \textbf{Herausgeber \newline
  (\zhs{主编})} &
  Deutsch-Chinesische Juristenvereinigung e.V. \newline
  Dr. Joachim Glatter, Präsident \newline
  E-Mail: \email{glatter@dcjv.org} & 
  ISSN: 1613-5768 \newline Online ISSN: 2366-7125 \\ 
  
  \textbf{Schriftleitung \newline
  (\zhs{执行编辑})} &
  Prof. Dr. Knut Benjamin Pißler, M.A. \newline
  Deutsch-Chinesisches Institut für Rechtswissen- schaft der Universitäten Göttingen und Nanjing \newline
  Hankou Lu 22, 210093 Nanjing, VR China & \\
  
   & \zhs{南京大学中德法学研究所} \newline
  \zhs{汉口路22号,} 
  210093 \zhs{南京, 中华人民共和国} & \\
  
   & Tel.\,/\,Fax: +86 25 8663 7892 \newline
  E-Mail: \email{dcir.nanjing@outlook.de} & \\
  
  \textbf{Wissenschaftlicher
  Beirat (\zhs{编委会})} &
  Prof. Dr. Björn Ahl, Professor für chinesische Rechtskultur, Universität zu Köln & \\
  
  \textbf{Online-Redaktion \newline
  (\zhs{电子版编辑部})} & 
  
  Max-Planck-Institut für ausländisches und internationales Privatrecht \newline
  Mittelweg 187, 20148 Hamburg & \\
  
   & Kontakt bei technischen Fragen: \newline David Schröder-Micheel \newline
  E-Mail: \email{micheel@mpipriv.de} & \\
  
  \textbf{Deutsches Korrek- \newline torat (\zhs{德语校对})} &
  Anja Rosenthal, Max-Planck-Institut für ausländisches und internationales Privatrecht & \\
  
  \textbf{Englisches Lek- \newline torat (\zhs{英语编辑})} &
  Michael Friedman, Max-Planck-Institut für ausländisches und internationales Privatrecht & \\
  
  \textbf{Gestaltung \newline
  (\zhs{美术设计})} &
  Jasper Habicht, Köln & \\
  
   & \imprintspan{Die Zeitschrift für Chinesisches Recht (ZChinR) erscheint viermal im Jahr als gedruckte Ausgabe. Das Abonnement der Zeitschrift ist für die Mitglieder der DCJV im Mitgliedsbeitrag enthalten. Es steht jedem Interessierten frei, Mitglied der DCJV zu werden. Eine Mitgliedschaft bei der Deutsch-Chinesischen Juristenvereinigung kann online unter \url{www.dcjv.org} beantragt werden.} \\
   
   & \imprintspan{Unter \url{www.ZChinR.de} stehen die Beiträge der jeweils vier letzten Ausgaben der Zeitschrift in Form von Inhaltsverzeichnissen, diejenigen der vorhergehenden Ausgaben als Volltexte im text- und seitenkonkordanten PDF-Format zur Verfügung. Mitglieder der DCJV können sich mit ihrem persönlichen Benutzernamen und Passwort anmelden und erhalten damit Zugriff auch auf die Volltexte der letzten vier Ausgaben.} \\
   
   & \imprintspan{Die Jahrgänge 1--10 (1994--2003) sind unter dem Titel ,,Newsletter der Deutsch-Chinesischen Juristenvereinigung e.V.`` erschienen. Die älteren Jahrgänge stehen im Internet unter \url{www.dcjv.de} im Volltext kostenfrei zum Abruf bereit.} \\
   
   & \imprintspan{Hinweise für Autoren finden sich unter derselben Adresse bei Unterpunkt ZChinR / Archiv.} \\
\end{imprint}

%EOF