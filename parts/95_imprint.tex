% BOF

\zchinrsetup{
  reset,
  article = {
    step chapter,
    page style   = {empty},
    title style  = {category},
    header style = {category},
    category     = {Impressum},
    pagination   = {none}
  }
}

\printtitle

\begin{imprint}
  \textbf{Herausgeber \newline
  (\zhs{主编})} &
  Deutsch-Chinesische Juristenvereinigung e.V. \newline
  Dr. Joachim Glatter, Präsident \newline
  E-Mail: \email{glatter@dcjv.org} & 
  ISSN: 1613-5768 \newline eISSN: 2366-7125 \\ 
  
  \textbf{Schriftleitung \newline
  (\zhs{执行编辑})} &
  Prof. Dr. Knut Benjamin Pißler, M.A. \newline
  Deutsch-Chinesisches Institut für Rechtswissenschaft der Universitäten Göttingen und Nanjing \newline
  Hankou Lu 22, 210093 Nanjing, VR China & 
  \ccsymbol[2] \ccbysymbol[2] \newline CC BY 4.0 \\
  
   & \zhs{南京大学中德法学研究所} \newline
  \zhs{汉口路22号,} 
  210093 \zhs{南京, 中华人民共和国} & \\
  
   & Tel.\,/\,Fax: +86 25 8663 7892 \newline
  E-Mail: \email{dcir.nanjing@outlook.de} & \\
  
  \textbf{Wissenschaftlicher
  Beirat (\zhs{编委会})} &
  Prof. Dr. Björn Ahl, Professor für chinesische Rechtskultur, Universität zu Köln & \\
   & Prof. Dr. Yuanshi Bu, Professorin für Internationales Wirtschaftsrecht mit Schwerpunkt Ostasien, Universität Freiburg & \\
   & Prof. Dr. (NTU) Georg Gesk, Professor für chinesisches Recht, Universität Osnabrück & \\

  \textbf{Internetpräsenz \newline
  (\zhs{刊物网站})} & 
  
  Max-Planck-Institut für ausländisches und internationales Privatrecht \newline
  Mittelweg 187, 20148 Hamburg & \\
  
   & Kontakt bei technischen Fragen: \newline David Schröder-Micheel \newline
  E-Mail: \email{micheel@mpipriv.de} & \\
  
  \textbf{Deutsches Korrek- \newline torat (\zhs{德语校对})} &
  Anja Rosenthal, Max-Planck-Institut für ausländisches und internationales Privatrecht & \\
  
  \textbf{Englisches Lek- \newline torat (\zhs{英语编辑})} &
  John Foulks und Michael Friedman, Max-Planck-Insti- tut für ausländisches und internationales Privatrecht & \\
  
  \textbf{Gestaltung \newline
  (\zhs{美术设计})} &
  Jasper Habicht, Köln & \\

  & \imprintline \\ 

  & \imprintspan{Die Zeitschrift für Chinesisches Recht (ZChinR) erscheint viermal im Jahr. Die Druckausgabe der Zeitschrift ist für die Mitglieder der DCJV im Mitgliedsbeitrag enthalten. Es steht allen Interessierten frei, Mitglied der DCJV zu werden. Eine Mitgliedschaft bei der Deutsch-Chinesischen Juristenvereinigung kann online unter \url{www.dcjv.org} beantragt werden.} \\

  & \imprintspan{Unter \url{www.ZChinR.org} stehen alle Beiträge seit 2004 als Volltexte im text- und seiten\-konkordanten PDF-Format zur Verfügung. Seit dem Jahr 2025 werden alle Beiträge unter der Creative-Commons-Lizenz CC\,BY\,4.0 veröffentlicht. Weitere Informationen zur Lizenz unter \url{www.creativecommons.org/licenses/by/4.0/}.} \\

  & \imprintspan{Die Jahrgänge 1--10 (1994--2003) sind unter dem Titel ,,Newsletter der Deutsch-Chinesischen Juristenvereinigung e.V.`` erschienen. Diese stehen im Internet unter \url{www.dcjv.org/publikationen/zchinr} im Volltext kostenfrei zum Abruf bereit.} \\ 
   
  & \imprintspan{Beiträge bitte stets auch in elektronischer Fassung an \email{zchinr@dcjv.org} als Word-Dokument einsenden. Weitere Hinweise für Autoren finden sich unter \url{www.ZChinR.org} beim Unterpunkt ,,Redaktionelle Hinweise``.} \\

\end{imprint}

%EOF