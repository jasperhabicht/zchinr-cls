% BOF

\zchinrsetup{
  reset,
  article = {
    doi          = {10.1080/10192557.2019.1651486},
    title style  = {title, author},
    title        = {Franz Kafka, Die Verwandlung, 1912.},
    author       = {Knut Benjamin Pißler},
    author note  = {Deutscher Vizedirektor am Deutsch-Chinesischen Institut für Rechtswissenschaft der Universitäten Göttingen und Nanjing, wissenschaftlicher Referent am Max-Planck-Institut für ausländisches und internationales Privatrecht in Hamburg (im Sabbatical) und Professor für chinesisches Recht an der Universität Göttingen.},
    toc style    = {author last},
    header style = {category},
    small title,
    two columns
  }
}

\printtitle

Als Gregor Samsa\footnote{\zhs{格里高尔·萨姆沙} (eine phonetische Übertragung des Namens) vom 1.1.2000, chinesisch-deutsch in diesem Heft, S.~111~ff.} eines Morgens aus unruhigen Träumen erwachte, fand er sich in seinem Bett zu einem ungeheueren Ungeziefer verwandelt. Er lag auf seinem panzerartig harten Rücken und sah, wenn er den Kopf ein wenig hob, seinen gewölbten, braunen, von bogenförmigen Versteifungen geteilten Bauch, auf dessen Höhe sich die Bettdecke, zum gänzlichen Niedergleiten bereit, kaum noch erhalten konnte. Seine vielen, im Vergleich zu seinem sonstigen Umfang kläglich dünnen Beine flimmerten ihm hilflos vor den Augen.

% EOF