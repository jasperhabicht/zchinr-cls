% !TeX program = xelatex
\documentclass[]{zchinr}

\zchinrsetup{
  color profile = {
    name = PSO Coated v3,
    info = PSOcoated_v3,
    path = PSOcoated_v3.icc
  },
  issue = {
    year   = 2025,
    number = 1
  }
}

\begin{document}

\zchinrsetup{
  article = {
    header style = {category},
    title style  = {category},
    category     = {Inhalt},
    pagination   = {roman}
  }
}

\printtitle

\printissuetoc


\zchinrsetup{
  reset,
  article = {
    doi          = {10.1080/10192557.2019.1651486},
    title        = {Beispiel für einen Aufsatz: Die Verwandlung als urheberrechtlich unproblematischer Beispieltext},
    title short  = {Beispiel für einen Aufsatz},
    title alt    = {Example for an Essay},
    author       = {Franz Kafka},
    author short = {Kafka},
    author note  = {Geboren am 3. Juli 1883 in Prag, Österreich-Ungarn; verstorben am 3. Juni 1924 in Kierling, Österreich; Schriftsteller.},
    abstract     = {Als Gregor Samsa eines Morgens aus unruhigen Träumen erwachte, fand er sich in seinem Bett zu einem ungeheuren Ungeziefer verwandelt.},
    abstract alt = {When Gregor Samsa woke up one morning from restless dreams, he found himself in his bed transformed into a monstrous vermin.},
    title style  = {category, title, author, abstract, contents},
    category     = {Aufsätze},
    two columns
  }
}

\printtitle

\section{I. Einleitung}

Text Als Gregor Samsa eines Morgens aus unruhigen Träumen erwachte, fand er sich in seinem Bett zu einem ungeheueren Ungeziefer verwandelt. Er lag auf seinem panzerartig harten Rücken und sah, wenn er den Kopf ein wenig hob, seinen gewölbten, braunen, von bogenförmigen Versteifungen geteilten Bauch, auf dessen Höhe sich die Bettdecke, zum gänzlichen Niedergleiten bereit, kaum noch erhalten konnte. Seine vielen, im Vergleich zu seinem sonstigen Umfang kläglich dünnen Beine flimmerten ihm hilflos vor den Augen.

,,Was ist mit mir geschehen?{}``, dachte er. Es war kein Traum. Sein Zimmer, ein richtiges, nur etwas zu kleines Menschenzimmer, lag ruhig zwischen den vier wohlbekannten Wänden. Über dem Tisch, auf dem eine auseinandergepackte Musterkollektion von Tuchwaren ausgebreitet war -- Samsa war Reisender -- hing das Bild, das er vor kurzem aus einer illustrierten Zeitschrift ausgeschnitten und in einem hübschen, vergoldeten Rahmen untergebracht hatte. Es stellte eine Dame dar, die mit einem Pelzhut und einer Pelzboa versehen, aufrecht dasaß und einen schweren Pelzmuff, in dem ihr ganzer Unterarm verschwunden war, dem Beschauer entgegenhob.

\subsection{1. Unterpunkt}

Gregors Blick richtete sich dann zum Fenster, und das trübe Wetter -- man hörte Regentropfen auf das Fensterblech aufschlagen -- machte ihn ganz melancholisch. ,,Wie wäre es, wenn ich noch ein wenig weiterschliefe und alle Narrheiten vergäße``, dachte er, aber das war gänzlich undurchführbar, denn er war gewöhnt, auf der rechten Seite zu schlafen, konnte sich aber in seinem gegenwärtigen Zustand nicht in diese Lage bringen. Mit welcher Kraft er sich auch auf die rechte Seite warf, immer wieder schaukelte er in die Rückenlage zurück. Er versuchte es wohl hundertmal, schloß die Augen, um die zappelnden Beine nicht sehen zu müssen, und ließ erst ab, als er in der Seite einen noch nie gefühlten, leichten, dumpfen Schmerz zu fühlen begann.

,,Ach Gott``, dachte er, ,,was für einen anstrengenden Beruf habe ich gewählt! Tag aus, Tag ein auf der Reise. Die geschäftlichen Aufregungen sind viel größer, als im eigentlichen Geschäft zu Hause, und außerdem ist mir noch diese Plage des Reisens auferlegt, die Sorgen um die Zuganschlüsse, das unregelmäßige, schlechte Essen, ein immer wechselnder, nie andauernder, nie herzlich werdender menschlicher Verkehr. Der Teufel soll das alles holen!{}`` Er fühlte ein leichtes Jucken oben auf dem Bauch; schob sich auf dem Rücken langsam näher zum Bettpfosten, um den Kopf besser heben zu können; fand die juckende Stelle, die mit lauter kleinen weißen Pünktchen besetzt war, die er nicht zu beurteilen verstand; und wollte mit einem Bein die Stelle betasten, zog es aber gleich zurück, denn bei der Berührung umwehten ihn Kälteschauer.

\subsection{2. Unerträglich langer Unterpunkt mit vielen Ausführungen, die über mehrere Zeilen reichen}

Er glitt wieder in seine frühere Lage zurück. ,,Dies frühzeitige Aufstehen``, dachte er, ,,macht einen ganz blödsinnig. Der Mensch muß seinen Schlaf haben. Andere Reisende leben wie Haremsfrauen. Wenn ich zum Beispiel im Laufe des Vormittags ins Gasthaus zurückgehe, um die erlangten Aufträge zu überschreiben, sitzen diese Herren erst beim Frühstück. Das sollte ich bei meinem Chef versuchen; ich würde auf der Stelle hinausfliegen. Wer weiß übrigens, ob das nicht sehr gut für mich wäre. Wenn ich mich nicht wegen meiner Eltern zurückhielte, ich hätte längst gekündigt, ich wäre vor den Chef hin getreten und hätte ihm meine Meinung von Grund des Herzens aus gesagt. Vom Pult hätte er fallen müssen! Es ist auch eine sonderbare Art, sich auf das Pult zu setzen und von der Höhe herab mit dem Angestellten zu reden, der überdies wegen der Schwerhörigkeit des Chefs ganz nahe herantreten muß. Nun, die Hoffnung ist noch nicht gänzlich aufgegeben; habe ich einmal das Geld beisammen, um die Schuld der Eltern an ihn abzuzahlen -- es dürfte noch fünf bis sechs Jahre dauern --, mache ich die Sache unbedingt. Dann wird der große Schnitt gemacht. Vorläufig allerdings muß ich aufstehen, denn mein Zug fährt um fünf.``

Und er sah zur Weckuhr hinüber, die auf dem Kasten tickte. ,,Himmlischer Vater!{}``, dachte er. Es war halb sieben Uhr, und die Zeiger gingen ruhig vorwärts, es war sogar halb vorüber, es näherte sich schon dreiviertel. Sollte der Wecker nicht geläutet haben? Man sah vom Bett aus, daß er auf vier Uhr richtig eingestellt war; gewiß hatte er auch geläutet. Ja, aber war es möglich, dieses möbelerschütternde Läuten ruhig zu verschlafen? Nun, ruhig hatte er ja nicht geschlafen, aber wahrscheinlich desto fester. Was aber sollte er jetzt tun? Der nächste Zug ging um sieben Uhr; um den einzuholen, hätte er sich unsinnig beeilen müssen, und die Kollektion war noch nicht eingepackt, und er selbst fühlte sich durchaus nicht besonders frisch und beweglich. Und selbst wenn er den Zug einholte, ein Donnerwetter des Chefs war nicht zu vermeiden, denn der Geschäftsdiener hatte beim Fünfuhrzug gewartet und die Meldung von seiner Versäumnis längst erstattet. Es war eine Kreatur des Chefs, ohne Rückgrat und Verstand. Wie nun, wenn er sich krank meldete? Das wäre aber äußerst peinlich und verdächtig, denn Gregor war während seines fünfjährigen Dienstes noch nicht einmal krank gewesen. Gewiß würde der Chef mit dem Krankenkassenarzt kommen, würde den Eltern wegen des faulen Sohnes Vorwürfe machen und alle Einwände durch den Hinweis auf den Krankenkassenarzt abschneiden, für den es ja überhaupt nur ganz gesunde, aber arbeitsscheue Menschen gibt. Und hätte er übrigens in diesem Falle so ganz unrecht? Gregor fühlte sich tatsächlich, abgesehen von einer nach dem langen Schlaf wirklich überflüssigen Schläfrigkeit, ganz wohl und hatte sogar einen besonders kräftigen Hunger.

\section{II. Fazit}

Als Gregor Samsa\footnote{\zhs{格里高尔·萨姆沙} (eine phonetische Übertragung des Namens) vom 1.1.2000, chinesisch-deutsch in diesem Heft, S.~111~ff.} eines Morgens aus unruhigen Träumen erwachte, fand er sich in seinem Bett zu einem ungeheueren Ungeziefer verwandelt. Er lag auf seinem panzerartig harten Rücken und sah, wenn er den Kopf ein wenig hob, seinen gewölbten, braunen, von bogenförmigen Versteifungen geteilten Bauch, auf dessen Höhe sich die Bettdecke, zum gänzlichen Niedergleiten bereit, kaum noch erhalten konnte. Seine vielen, im Vergleich zu seinem sonstigen Umfang kläglich dünnen Beine flimmerten ihm hilflos vor den Augen.

,,Was ist mit mir geschehen?{}``, dachte er. Es war kein Traum. Sein Zimmer, ein richtiges, nur etwas zu kleines Menschenzimmer, lag ruhig zwischen den vier wohlbekannten Wänden. Über dem Tisch, auf dem eine auseinandergepackte Musterkollektion von Tuchwaren ausgebreitet war -- Samsa war Reisender -- hing das Bild, das er vor kurzem aus einer illustrierten Zeitschrift ausgeschnitten und in einem hübschen, vergoldeten Rahmen untergebracht hatte. Es stellte eine Dame dar, die mit einem Pelzhut und einer Pelzboa versehen, aufrecht dasaß und einen schweren Pelzmuff, in dem ihr ganzer Unterarm verschwunden war, dem Beschauer entgegenhob.

Gregors Blick richtete sich dann zum Fenster, und das trübe Wetter -- man hörte Regentropfen auf das Fensterblech aufschlagen -- machte ihn ganz melancholisch. ,,Wie wäre es, wenn ich noch ein wenig weiterschliefe und alle Narrheiten vergäße``, dachte er, aber das war gänzlich undurchführbar, denn er war gewöhnt, auf der rechten Seite zu schlafen, konnte sich aber in seinem gegenwärtigen Zustand nicht in diese Lage bringen. Mit welcher Kraft er sich auch auf die rechte Seite warf, immer wieder schaukelte er in die Rückenlage zurück. Er versuchte es wohl hundertmal, schloß die Augen, um die zappelnden Beine nicht sehen zu müssen, und ließ erst ab, als er in der Seite einen noch nie gefühlten, leichten, dumpfen Schmerz zu fühlen begann.

,,Ach Gott``, dachte er, ,,was für einen anstrengenden Beruf habe ich gewählt! Tag aus, Tag ein auf der Reise. Die geschäftlichen Aufregungen sind viel größer, als im eigentlichen Geschäft zu Hause, und außerdem ist mir noch diese Plage des Reisens auferlegt, die Sorgen um die Zuganschlüsse, das unregelmäßige, schlechte Essen, ein immer wechselnder, nie andauernder, nie herzlich werdender menschlicher Verkehr. Der Teufel soll das alles holen!{}`` Er fühlte ein leichtes Jucken oben auf dem Bauch; schob sich auf dem Rücken langsam näher zum Bettpfosten, um den Kopf besser heben zu können; fand die juckende Stelle, die mit lauter kleinen weißen Pünktchen besetzt war, die er nicht zu beurteilen verstand; und wollte mit einem Bein die Stelle betasten, zog es aber gleich zurück, denn bei der Berührung umwehten ihn Kälteschauer.

Er glitt wieder in seine frühere Lage zurück. ,,Dies frühzeitige Aufstehen``, dachte er, ,,macht einen ganz blödsinnig. Der Mensch muß seinen Schlaf haben. Andere Reisende leben wie Haremsfrauen. Wenn ich zum Beispiel im Laufe des Vormittags ins Gasthaus zurückgehe, um die erlangten Aufträge zu überschreiben, sitzen diese Herren erst beim Frühstück. Das sollte ich bei meinem Chef versuchen; ich würde auf der Stelle hinausfliegen. Wer weiß übrigens, ob das nicht sehr gut für mich wäre. Wenn ich mich nicht wegen meiner Eltern zurückhielte, ich hätte längst gekündigt, ich wäre vor den Chef hin getreten und hätte ihm meine Meinung von Grund des Herzens aus gesagt. Vom Pult hätte er fallen müssen! Es ist auch eine sonderbare Art, sich auf das Pult zu setzen und von der Höhe herab mit dem Angestellten zu reden, der überdies wegen der Schwerhörigkeit des Chefs ganz nahe herantreten muß. Nun, die Hoffnung ist noch nicht gänzlich aufgegeben; habe ich einmal das Geld beisammen, um die Schuld der Eltern an ihn abzuzahlen -- es dürfte noch fünf bis sechs Jahre dauern --, mache ich die Sache unbedingt. Dann wird der große Schnitt gemacht. Vorläufig allerdings muß ich aufstehen, denn mein Zug fährt um fünf.``

Und er sah zur Weckuhr hinüber, die auf dem Kasten tickte. ,,Himmlischer Vater!{}``, dachte er. Es war halb sieben Uhr, und die Zeiger gingen ruhig vorwärts, es war sogar halb vorüber, es näherte sich schon dreiviertel. Sollte der Wecker nicht geläutet haben? Man sah vom Bett aus, daß er auf vier Uhr richtig eingestellt war; gewiß hatte er auch geläutet. Ja, aber war es möglich, dieses möbelerschütternde Läuten ruhig zu verschlafen? Nun, ruhig hatte er ja nicht geschlafen, aber wahrscheinlich desto fester. Was aber sollte er jetzt tun? Der nächste Zug ging um sieben Uhr; um den einzuholen, hätte er sich unsinnig beeilen müssen, und die Kollektion war noch nicht eingepackt, und er selbst fühlte sich durchaus nicht besonders frisch und beweglich. Und selbst wenn er den Zug einholte, ein Donnerwetter des Chefs war nicht zu vermeiden, denn der Geschäftsdiener hatte beim Fünfuhrzug gewartet und die Meldung von seiner Versäumnis längst erstattet. Es war eine Kreatur des Chefs, ohne Rückgrat und Verstand. Wie nun, wenn er sich krank meldete? Das wäre aber äußerst peinlich und verdächtig, denn Gregor war während seines fünfjährigen Dienstes noch nicht einmal krank gewesen. Gewiß würde der Chef mit dem Krankenkassenarzt kommen, würde den Eltern wegen des faulen Sohnes Vorwürfe machen und alle Einwände durch den Hinweis auf den Krankenkassenarzt abschneiden, für den es ja überhaupt nur ganz gesunde, aber arbeitsscheue Menschen gibt. Und hätte er übrigens in diesem Falle so ganz unrecht? Gregor fühlte sich tatsächlich, abgesehen von einer nach dem langen Schlaf wirklich überflüssigen Schläfrigkeit, ganz wohl und hatte sogar einen besonders kräftigen Hunger.


\zchinrsetup{
  reset,
  article = {
    doi          = {10.1080/10192557.2019.1651486},
    title        = {Beispiel für einen kurzen Beitrag: Die Verwandlung als urheberrechtlich unproblematischer Beispieltext},
    title short  = {Beispiel für einen kurzen Beitrag},
    title alt    = {Example for a Short Article},
    author       = {Franz Kafka},
    author short = {Kafka},
    author note  = {Geboren am 3. Juli 1883 in Prag, Österreich-Ungarn; verstorben am 3. Juni 1924 in Kierling, Österreich; Schriftsteller.},
    abstract     = {Als Gregor Samsa eines Morgens aus unruhigen Träumen erwachte, fand er sich in seinem Bett zu einem ungeheuren Ungeziefer verwandelt.},
    abstract alt = {When Gregor Samsa woke up one morning from restless dreams, he found himself in his bed transformed into a monstrous vermin.},
    title style  = {category, title, abstract, author},
    category     = {Kurze Beiträge},
    two columns
  }
}

\printtitle

\section{I. Einleitung}

Als Gregor Samsa\footnote{\zhs{格里高尔·萨姆沙} (eine phonetische Übertragung des Namens) vom 1.1.2000, chinesisch-deutsch in diesem Heft, S.~111~ff.} eines Morgens aus unruhigen Träumen erwachte, fand er sich in seinem Bett zu einem ungeheueren Ungeziefer verwandelt. Er lag auf seinem panzerartig harten Rücken und sah, wenn er den Kopf ein wenig hob, seinen gewölbten, braunen, von bogenförmigen Versteifungen geteilten Bauch, auf dessen Höhe sich die Bettdecke, zum gänzlichen Niedergleiten bereit, kaum noch erhalten konnte. Seine vielen, im Vergleich zu seinem sonstigen Umfang kläglich dünnen Beine flimmerten ihm hilflos vor den Augen.

,,Was ist mit mir geschehen?{}``, dachte er. Es war kein Traum. Sein Zimmer, ein richtiges, nur etwas zu kleines Menschenzimmer, lag ruhig zwischen den vier wohlbekannten Wänden. Über dem Tisch, auf dem eine auseinandergepackte Musterkollektion von Tuchwaren ausgebreitet war -- Samsa war Reisender -- hing das Bild, das er vor kurzem aus einer illustrierten Zeitschrift ausgeschnitten und in einem hübschen, vergoldeten Rahmen untergebracht hatte. Es stellte eine Dame dar, die mit einem Pelzhut und einer Pelzboa versehen, aufrecht dasaß und einen schweren Pelzmuff, in dem ihr ganzer Unterarm verschwunden war, dem Beschauer entgegenhob.\footnote{\url{https://pkulaw.cn}}

\ldots{}


\zchinrsetup{
  reset,
  article = {
    doi          = {10.1080/10192557.2019.1651486},
    title        = {Verwandlungsgesetz der VR China},
    title short  = {Verwandlungsgesetz},
    author       = {Knut Benjamin Pißler},
    author short = {Pißler},
    title style  = {category, title},
    header style = {title},
    toc style    = {author last},
    category     = {Dokumentationen}
  }
}

\printtitle

\begin{documentation}

\documentationheader[c]{\zhs{中华人民共和国变形法}} & \documentationheader[c]{Verwandlungsgesetz der VR China} \\

\documentationheader[c][n]{\zhs{(零号)}} & \documentationheader[c][n]{(Nr. 0))} \\

\documentationheader[l]{\zhs{第一章 一天早上,当格里高尔·萨姆沙从不安的梦中醒来时}} & \documentationheader[l]{1. Kapitel: Als Gregor Samsa eines Morgens aus unruhigen Träumen erwachte}[section] \\

\zhs{\textbf{第一条} 一天早上,当格里高尔·萨姆沙从不安的梦中醒来时,他发现自己变成了床上的一只畸形害虫。他仰面躺在坚硬如盔甲的背上,稍微抬起头,就能看到自己隆起的棕色腹部,腹部被拱形的绷带分割开来,羽绒被就在绷带的高度,随时都有可能完全滑落,但却几乎无法支撑住自己。他的多条腿与平时的粗壮相比瘦得可怜,无助地在他眼前晃动。} & \textbf{§~1} Als Gregor Samsa\footnote{\zhs{格里高尔·萨姆沙} (eine phonetische Übertragung des Namens) vom 1.1.2000, chinesisch-deutsch in diesem Heft, S.~111~ff.} eines Morgens aus unruhigen Träumen erwachte, fand er sich in seinem Bett zu einem ungeheueren Ungeziefer verwandelt. Er lag auf seinem panzerartig harten Rücken und sah, wenn er den Kopf ein wenig hob, seinen gewölbten, braunen, von bogenförmigen Versteifungen geteilten Bauch, auf dessen Höhe sich die Bettdecke, zum gänzlichen Niedergleiten bereit, kaum noch erhalten konnte. Seine vielen, im Vergleich zu seinem sonstigen Umfang kläglich dünnen Beine flimmerten ihm hilflos vor den Augen. \\

\zhs{\textbf{第二条} 一天早上,当格里高尔·萨姆沙从不安的梦中醒来时,他发现自己变成了床上的一只畸形害虫。} & \textbf{§~2} Als Gregor Samsa\footnote{\zhs{格里高尔·萨姆沙} (eine phonetische Übertragung des Namens) vom 1.1.2000, chinesisch-deutsch in diesem Heft, S.~111~ff.} eines Morgens aus unruhigen Träumen erwachte, fand er sich in seinem Bett zu einem ungeheueren Ungeziefer verwandelt. \\

\zhs{\textbf{第三条} 一天早上,当格里高尔·萨姆沙从不安的梦中醒来时,他发现自己变成了床上的一只畸形害虫。} & \textbf{§~3} Als Gregor Samsa\footnote{\zhs{格里高尔·萨姆沙} (eine phonetische Übertragung des Namens) vom 1.1.2000, chinesisch-deutsch in diesem Heft, S.~111~ff.} eines Morgens aus unruhigen Träumen erwachte, fand er sich in seinem Bett zu einem ungeheueren Ungeziefer verwandelt. \\

\zhs{\textbf{第四条} 一天早上,当格里高尔·萨姆沙从不安的梦中醒来时,他发现自己变成了床上的一只畸形害虫。} & \textbf{§~4} Als Gregor Samsa\footnote{\zhs{格里高尔·萨姆沙} (eine phonetische Übertragung des Namens) vom 1.1.2000, chinesisch-deutsch in diesem Heft, S.~111~ff.} eines Morgens aus unruhigen Träumen erwachte, fand er sich in seinem Bett zu einem ungeheueren Ungeziefer verwandelt. \\

\zhs{\textbf{第五条} 一天早上,当格里高尔·萨姆沙从不安的梦中醒来时,他发现自己变成了床上的一只畸形害虫。} & \textbf{§~5} Als Gregor Samsa\footnote{\zhs{格里高尔·萨姆沙} (eine phonetische Übertragung des Namens) vom 1.1.2000, chinesisch-deutsch in diesem Heft, S.~111~ff.} eines Morgens aus unruhigen Träumen erwachte, fand er sich in seinem Bett zu einem ungeheueren Ungeziefer verwandelt. \\

\zhs{\textbf{第六条} 他又悄悄地回到了原来的位置。“他想,"起得这么早,让你变得很愚蠢。一个人必须有他的睡眠。其他旅行者的生活就像后宫女人。比如,当我早上回到客栈签收订单时,这些先生们只坐下来吃早餐。我应该跟我的老板试试,我肯定会被当场赶出去。顺便说一句,谁知道这对我是不是很有好处。如果不是因为父母,我早就辞职了,我会站在老板面前,发自内心地告诉他我的想法。他肯定会从办公桌上摔下来!坐在办公桌上从高处跟员工说话也是一种很奇怪的方式,因为老板听力不好,还得靠得很近。好吧,我还没有完全放弃希望;一旦我凑齐了钱,还清了我父母欠他的债--这应该还需要五六年的时间--我肯定会这么做的。然后再大刀阔斧地干。不过,现在我得起床了,因为我的火车五点钟就要开了。} & \textbf{§~6} Er glitt wieder in seine frühere Lage zurück. ,,Dies frühzeitige Aufstehen``, dachte er, ,,macht einen ganz blödsinnig. Der Mensch muß seinen Schlaf haben. Andere Reisende leben wie Haremsfrauen. Wenn ich zum Beispiel im Laufe des Vormittags ins Gasthaus zurückgehe, um die erlangten Aufträge zu überschreiben, sitzen diese Herren erst beim Frühstück. Das sollte ich bei meinem Chef versuchen; ich würde auf der Stelle hinausfliegen. Wer weiß übrigens, ob das nicht sehr gut für mich wäre. Wenn ich mich nicht wegen meiner Eltern zurückhielte, ich hätte längst gekündigt, ich wäre vor den Chef hin getreten und hätte ihm meine Meinung von Grund des Herzens aus gesagt. Vom Pult hätte er fallen müssen! Es ist auch eine sonderbare Art, sich auf das Pult zu setzen und von der Höhe herab mit dem Angestellten zu reden, der überdies wegen der Schwerhörigkeit des Chefs ganz nahe herantreten muß. Nun, die Hoffnung ist noch nicht gänzlich aufgegeben; habe ich einmal das Geld beisammen, um die Schuld der Eltern an ihn abzuzahlen -- es dürfte noch fünf bis sechs Jahre dauern --, mache ich die Sache unbedingt. Dann wird der große Schnitt gemacht. Vorläufig allerdings muß ich aufstehen, denn mein Zug fährt um fünf.`` \\

\zhs{\textbf{第七条} 他看了看盒子上滴答作响的闹钟。“天父啊,"他想。六点半了,指针还在安静地滴答作响,甚至已经过了一半,已经快到四分之三了。闹钟不是应该响了吗?从床上可以看出,闹钟正确地定在了四点钟,它肯定已经响了。是的,但有可能在这震撼家具的铃声中安静地入睡吗?他睡得并不安稳,但可能更安稳。但他现在该怎么办呢?} & \textbf{§~7} Und er sah zur Weckuhr hinüber, die auf dem Kasten tickte. ,,Himmlischer Vater!{}``, dachte er. Es war halb sieben Uhr, und die Zeiger gingen ruhig vorwärts, es war sogar halb vorüber, es näherte sich schon dreiviertel. Sollte der Wecker nicht geläutet haben? Man sah vom Bett aus, daß er auf vier Uhr richtig eingestellt war; gewiß hatte er auch geläutet. Ja, aber war es möglich, dieses möbelerschütternde Läuten ruhig zu verschlafen? Nun, ruhig hatte er ja nicht geschlafen, aber wahrscheinlich desto fester. Was aber sollte er jetzt tun? \\

\zhs{一) 下一班火车七点钟发车,他必须匆匆忙忙地赶上去,} &
1) Der nächste Zug ging um sieben Uhr; um den einzuholen, hätte er sich unsinnig beeilen müssen, \\

\zhs{二) 而且行李还没有收拾好,他自己也不觉得特别清爽,行动也不方便。} &
2) und die Kollektion war noch nicht eingepackt, und er selbst fühlte sich durchaus nicht besonders frisch und beweglich. \\

\zhs{三) 即使他赶上了火车,也无法避免老板的雷雨,} &
3) Und selbst wenn er den Zug einholte, ein Donnerwetter des Chefs war nicht zu vermeiden, \\

\zhs{四) 因为男仆已经在五点钟的火车上等着了,而且早就报告了他的延误。} & 
4) denn der Geschäftsdiener hatte beim Fünfuhrzug gewartet und die Meldung von seiner Versäumnis längst erstattet. \\

\\

& Übersetungen: DeepL, \url{https://www.deepl.com}. \\
\end{documentation}


\zchinrsetup{
  reset,
  article = {
    doi          = {10.1080/10192557.2019.1651486},
    title style  = {category, title, author},
    title        = {Franz Kafka, Die Verwandlung, 1912.},
    author       = {Knut Benjamin Pißler},
    category     = {Rezensionen},
    toc style    = {author last},
    header style = {category},
    small title,
    two columns
  }
}

\printtitle

Als Gregor Samsa\footnote{\zhs{格里高尔·萨姆沙} (eine phonetische Übertragung des Namens) vom 1.1.2000, chinesisch-deutsch in diesem Heft, S.~111~ff.} eines Morgens aus unruhigen Träumen erwachte, fand er sich in seinem Bett zu einem ungeheueren Ungeziefer verwandelt. Er lag auf seinem panzerartig harten Rücken und sah, wenn er den Kopf ein wenig hob, seinen gewölbten, braunen, von bogenförmigen Versteifungen geteilten Bauch, auf dessen Höhe sich die Bettdecke, zum gänzlichen Niedergleiten bereit, kaum noch erhalten konnte. Seine vielen, im Vergleich zu seinem sonstigen Umfang kläglich dünnen Beine flimmerten ihm hilflos vor den Augen.

,,Was ist mit mir geschehen?{}``, dachte er. Es war kein Traum. Sein Zimmer, ein richtiges, nur etwas zu kleines Menschenzimmer, lag ruhig zwischen den vier wohlbekannten Wänden. Über dem Tisch, auf dem eine auseinandergepackte Musterkollektion von Tuchwaren ausgebreitet war -- Samsa war Reisender -- hing das Bild, das er vor kurzem aus einer illustrierten Zeitschrift ausgeschnitten und in einem hübschen, vergoldeten Rahmen untergebracht hatte. Es stellte eine Dame dar, die mit einem Pelzhut und einer Pelzboa versehen, aufrecht dasaß und einen schweren Pelzmuff, in dem ihr ganzer Unterarm verschwunden war, dem Beschauer entgegenhob.

Gregors Blick richtete sich dann zum Fenster, und das trübe Wetter -- man hörte Regentropfen auf das Fensterblech aufschlagen -- machte ihn ganz melancholisch. ,,Wie wäre es, wenn ich noch ein wenig weiterschliefe und alle Narrheiten vergäße``, dachte er, aber das war gänzlich undurchführbar, denn er war gewöhnt, auf der rechten Seite zu schlafen, konnte sich aber in seinem gegenwärtigen Zustand nicht in diese Lage bringen. Mit welcher Kraft er sich auch auf die rechte Seite warf, immer wieder schaukelte er in die Rückenlage zurück. Er versuchte es wohl hundertmal, schloß die Augen, um die zappelnden Beine nicht sehen zu müssen, und ließ erst ab, als er in der Seite einen noch nie gefühlten, leichten, dumpfen Schmerz zu fühlen begann.

,,Ach Gott``, dachte er, ,,was für einen anstrengenden Beruf habe ich gewählt! Tag aus, Tag ein auf der Reise. Die geschäftlichen Aufregungen sind viel größer, als im eigentlichen Geschäft zu Hause, und außerdem ist mir noch diese Plage des Reisens auferlegt, die Sorgen um die Zuganschlüsse, das unregelmäßige, schlechte Essen, ein immer wechselnder, nie andauernder, nie herzlich werdender menschlicher Verkehr. Der Teufel soll das alles holen!{}`` Er fühlte ein leichtes Jucken oben auf dem Bauch; schob sich auf dem Rücken langsam näher zum Bettpfosten, um den Kopf besser heben zu können; fand die juckende Stelle, die mit lauter kleinen weißen Pünktchen besetzt war, die er nicht zu beurteilen verstand; und wollte mit einem Bein die Stelle betasten, zog es aber gleich zurück, denn bei der Berührung umwehten ihn Kälteschauer.

Er glitt wieder in seine frühere Lage zurück. ,,Dies frühzeitige Aufstehen``, dachte er, ,,macht einen ganz blödsinnig. Der Mensch muß seinen Schlaf haben. Andere Reisende leben wie Haremsfrauen. Wenn ich zum Beispiel im Laufe des Vormittags ins Gasthaus zurückgehe, um die erlangten Aufträge zu überschreiben, sitzen diese Herren erst beim Frühstück. Das sollte ich bei meinem Chef versuchen; ich würde auf der Stelle hinausfliegen. Wer weiß übrigens, ob das nicht sehr gut für mich wäre. Wenn ich mich nicht wegen meiner Eltern zurückhielte, ich hätte längst gekündigt, ich wäre vor den Chef hin getreten und hätte ihm meine Meinung von Grund des Herzens aus gesagt. Vom Pult hätte er fallen müssen! Es ist auch eine sonderbare Art, sich auf das Pult zu setzen und von der Höhe herab mit dem Angestellten zu reden, der überdies wegen der Schwerhörigkeit des Chefs ganz nahe herantreten muß. Nun, die Hoffnung ist noch nicht gänzlich aufgegeben; habe ich einmal das Geld beisammen, um die Schuld der Eltern an ihn abzuzahlen -- es dürfte noch fünf bis sechs Jahre dauern --, mache ich die Sache unbedingt. Dann wird der große Schnitt gemacht. Vorläufig allerdings muß ich aufstehen, denn mein Zug fährt um fünf.``

Und er sah zur Weckuhr hinüber, die auf dem Kasten tickte. ,,Himmlischer Vater!{}``, dachte er. Es war halb sieben Uhr, und die Zeiger gingen ruhig vorwärts, es war sogar halb vorüber, es näherte sich schon dreiviertel. Sollte der Wecker nicht geläutet haben? Man sah vom Bett aus, daß er auf vier Uhr richtig eingestellt war; gewiß hatte er auch geläutet. Ja, aber war es möglich, dieses möbelerschütternde Läuten ruhig zu verschlafen? Nun, ruhig hatte er ja nicht geschlafen, aber wahrscheinlich desto fester. Was aber sollte er jetzt tun? Der nächste Zug ging um sieben Uhr; um den einzuholen, hätte er sich unsinnig beeilen müssen, und die Kollektion war noch nicht eingepackt, und er selbst fühlte sich durchaus nicht besonders frisch und beweglich. Und selbst wenn er den Zug einholte, ein Donnerwetter des Chefs war nicht zu vermeiden, denn der Geschäftsdiener hatte beim Fünfuhrzug gewartet und die Meldung von seiner Versäumnis längst erstattet. Es war eine Kreatur des Chefs, ohne Rückgrat und Verstand. Wie nun, wenn er sich krank meldete? Das wäre aber äußerst peinlich und verdächtig, denn Gregor war während seines fünfjährigen Dienstes noch nicht einmal krank gewesen. Gewiß würde der Chef mit dem Krankenkassenarzt kommen, würde den Eltern wegen des faulen Sohnes Vorwürfe machen und alle Einwände durch den Hinweis auf den Krankenkassenarzt abschneiden, für den es ja überhaupt nur ganz gesunde, aber arbeitsscheue Menschen gibt. Und hätte er übrigens in diesem Falle so ganz unrecht? Gregor fühlte sich tatsächlich, abgesehen von einer nach dem langen Schlaf wirklich überflüssigen Schläfrigkeit, ganz wohl und hatte sogar einen besonders kräftigen Hunger.

Als Gregor Samsa eines Morgens aus unruhigen Träumen erwachte, fand er sich in seinem Bett zu einem ungeheueren Ungeziefer verwandelt. Er lag auf seinem panzerartig harten Rücken und sah, wenn er den Kopf ein wenig hob, seinen gewölbten, braunen, von bogenförmigen Versteifungen geteilten Bauch, auf dessen Höhe sich die Bettdecke, zum gänzlichen Niedergleiten bereit, kaum noch erhalten konnte. Seine vielen, im Vergleich zu seinem sonstigen Umfang kläglich dünnen Beine flimmerten ihm hilflos vor den Augen.

,,Was ist mit mir geschehen?{}``, dachte er. Es war kein Traum. Sein Zimmer, ein richtiges, nur etwas zu kleines Menschenzimmer, lag ruhig zwischen den vier wohlbekannten Wänden. Über dem Tisch, auf dem eine auseinandergepackte Musterkollektion von Tuchwaren ausgebreitet war -- Samsa war Reisender -- hing das Bild, das er vor kurzem aus einer illustrierten Zeitschrift ausgeschnitten und in einem hübschen, vergoldeten Rahmen untergebracht hatte. Es stellte eine Dame dar, die mit einem Pelzhut und einer Pelzboa versehen, aufrecht dasaß und einen schweren Pelzmuff, in dem ihr ganzer Unterarm verschwunden war, dem Beschauer entgegenhob.

Gregors Blick richtete sich dann zum Fenster, und das trübe Wetter -- man hörte Regentropfen auf das Fensterblech aufschlagen -- machte ihn ganz melancholisch. ,,Wie wäre es, wenn ich noch ein wenig weiterschliefe und alle Narrheiten vergäße``, dachte er, aber das war gänzlich undurchführbar, denn er war gewöhnt, auf der rechten Seite zu schlafen, konnte sich aber in seinem gegenwärtigen Zustand nicht in diese Lage bringen. Mit welcher Kraft er sich auch auf die rechte Seite warf, immer wieder schaukelte er in die Rückenlage zurück. Er versuchte es wohl hundertmal, schloß die Augen, um die zappelnden Beine nicht sehen zu müssen, und ließ erst ab, als er in der Seite einen noch nie gefühlten, leichten, dumpfen Schmerz zu fühlen begann.

,,Ach Gott``, dachte er, ,,was für einen anstrengenden Beruf habe ich gewählt! Tag aus, Tag ein auf der Reise. Die geschäftlichen Aufregungen sind viel größer, als im eigentlichen Geschäft zu Hause, und außerdem ist mir noch diese Plage des Reisens auferlegt, die Sorgen um die Zuganschlüsse, das unregelmäßige, schlechte Essen, ein immer wechselnder, nie andauernder, nie herzlich werdender menschlicher Verkehr. Der Teufel soll das alles holen!{}`` Er fühlte ein leichtes Jucken oben auf dem Bauch; schob sich auf dem Rücken langsam näher zum Bettpfosten, um den Kopf besser heben zu können; fand die juckende Stelle, die mit lauter kleinen weißen Pünktchen besetzt war, die er nicht zu beurteilen verstand; und wollte mit einem Bein die Stelle betasten, zog es aber gleich zurück, denn bei der Berührung umwehten ihn Kälteschauer.


\zchinrsetup{
  reset,
  article = {
    doi          = {10.1080/10192557.2019.1651486},
    title style  = {title, author},
    title        = {Franz Kafka, Die Verwandlung, 1912.},
    author       = {Knut Benjamin Pißler},
    toc style    = {author last},
    header style = {category},
    small title,
    two columns
  }
}

\printtitle

Als Gregor Samsa\footnote{\zhs{格里高尔·萨姆沙} (eine phonetische Übertragung des Namens) vom 1.1.2000, chinesisch-deutsch in diesem Heft, S.~111~ff.} eines Morgens aus unruhigen Träumen erwachte, fand er sich in seinem Bett zu einem ungeheueren Ungeziefer verwandelt. Er lag auf seinem panzerartig harten Rücken und sah, wenn er den Kopf ein wenig hob, seinen gewölbten, braunen, von bogenförmigen Versteifungen geteilten Bauch, auf dessen Höhe sich die Bettdecke, zum gänzlichen Niedergleiten bereit, kaum noch erhalten konnte. Seine vielen, im Vergleich zu seinem sonstigen Umfang kläglich dünnen Beine flimmerten ihm hilflos vor den Augen.


\zchinrsetup{
  reset,
  article = {
    doi          = {10.1080/10192557.2019.1651486},
    title style  = {category},
    header style = {category},
    category     = {Adressen}
  }
}

\printtitle

\addressessection{Beijing} 

\begin{addresses}
  \addressesheader{Moumou Kanzlei} & \addressesheader{\zhs{某某律师事务所}} \\
  Suite 4711, Moumou Building \newline 
  No. 1, Moumou Avenue \newline 
  100000 Beijing, VR China & 
  \zhs{某某大厦 4711 室} \newline 
  \zhs{某某大街 1 号} \newline
  \zhs{100000 北京, 中华人民共和国} \\
  \addressesspan{Tel.: +86 10 1234 5678 \addressessep Fax: +86 10 1234 6789 \newline 
  E-Mail: \email{franz.kafka@verwandlung.com} \newline
  Ansprechpartner: \emph{Franz Kafka}} \\
  
  \addressesheader{Moumou Kanzlei} & \addressesheader{\zhs{某某律师事务所}} \\
  Suite 4711, Moumou Building \newline 
  No. 1, Moumou Avenue \newline 
  100000 Beijing, VR China & 
  \zhs{某某大厦 4711 室} \newline 
  \zhs{某某大街 1 号} \newline
  \zhs{100000 北京, 中华人民共和国} \\
  \addressesspan{Tel.: +86 10 1234 5678 \addressessep Fax: +86 10 1234 6789 \newline 
  E-Mail: \email{franz.kafka@verwandlung.com} \newline
  Ansprechpartner: \emph{Franz Kafka}} \\
\end{addresses}

\addressessection{Shanghai}

\begin{addresses}
  \addressesheader{Moumou Kanzlei} & \addressesheader{\zhs{某某律师事务所}} \\
  Suite 4711, Moumou Building \newline 
  No. 1, Moumou Avenue \newline 
  200000 Shanghai, VR China & 
  \zhs{某某大厦 4711 室} \newline 
  \zhs{某某大街 1 号} \newline
  \zhs{200000 上海, 中华人民共和国} \\
  \addressesspan{Tel.: +86 20 1234 5678 \addressessep Fax: +86 20 1234 6789 \newline 
  E-Mail: \email{franz.kafka@verwandlung.com} \newline
  Ansprechpartner: \emph{Franz Kafka}} \\
\end{addresses}


\zchinrsetup{
  reset,
  article = {
    title style  = {category},
    header style = {category},
    category     = {Impressum}
  }
}

\printtitle

\begin{imprint}
  \textbf{Herausgeber \newline
  (\zhs{主编})} &
  Deutsch-Chinesische Juristenvereinigung e.V. \newline
  Dr. Joachim Glatter, Präsident \newline
  E-Mail: \email{glatter@dcjv.org} & 
  ISSN: 1613-5768 \newline Online ISSN: 2366-7125 \\ 
  
  \textbf{Schriftleitung \newline
  (\zhs{执行编辑})} &
  Prof. Dr. Knut Benjamin Pißler, M.A. \newline
  Deutsch-Chinesisches Institut für Rechtswissen- schaft der Universitäten Göttingen und Nanjing \newline
  Hankou Lu 22, 210093 Nanjing, VR China & \\
  
   & \zhs{南京大学中德法学研究所} \newline
  \zhs{汉口路22号,} 
  210093 \zhs{南京, 中华人民共和国} & \\
  
   & Tel.\,/\,Fax: +86 25 8663 7892 \newline
  E-Mail: \email{dcir.nanjing@outlook.de} & \\
  
  \textbf{Wissenschaftlicher
  Beirat (\zhs{编委会})} &
  Prof. Dr. Björn Ahl, Professor für chinesische Rechtskultur, Universität zu Köln & \\
  
  \textbf{Online-Redaktion \newline
  (\zhs{电子版编辑部})} & 
  
  Max-Planck-Institut für ausländisches und internationales Privatrecht \newline
  Mittelweg 187, 20148 Hamburg & \\
  
   & Kontakt bei technischen Fragen: \newline David Schröder-Micheel \newline
  E-Mail: \email{micheel@mpipriv.de} & \\
  
  \textbf{Deutsches Korrek- \newline torat (\zhs{德语校对})} &
  Anja Rosenthal, Max-Planck-Institut für ausländisches und internationales Privatrecht & \\
  
  \textbf{Englisches Lek- \newline torat (\zhs{英语编辑})} &
  Michael Friedman, Max-Planck-Institut für ausländisches und internationales Privatrecht & \\
  
  \textbf{Gestaltung \newline
  (\zhs{美术设计})} &
  Jasper Habicht, Köln & \\
  
   & \imprintspan{Die Zeitschrift für Chinesisches Recht (ZChinR) erscheint viermal im Jahr als gedruckte Ausgabe. Das Abonnement der Zeitschrift ist für die Mitglieder der DCJV im Mitgliedsbeitrag enthalten. Es steht jedem Interessierten frei, Mitglied der DCJV zu werden. Eine Mitgliedschaft bei der Deutsch-Chinesischen Juristenvereinigung kann online unter \url{www.dcjv.org} beantragt werden.} \\
   
   & \imprintspan{Unter \url{www.ZChinR.de} stehen die Beiträge der jeweils vier letzten Ausgaben der Zeitschrift in Form von Inhaltsverzeichnissen, diejenigen der vorhergehenden Ausgaben als Volltexte im text- und seitenkonkordanten PDF-Format zur Verfügung. Mitglieder der DCJV können sich mit ihrem persönlichen Benutzernamen und Passwort anmelden und erhalten damit Zugriff auch auf die Volltexte der letzten vier Ausgaben.} \\
   
   & \imprintspan{Die Jahrgänge 1--10 (1994--2003) sind unter dem Titel ,,Newsletter der Deutsch-Chinesischen Juristenvereinigung e.V.`` erschienen. Die älteren Jahrgänge stehen im Internet unter \url{www.dcjv.de} im Volltext kostenfrei zum Abruf bereit.} \\
   
   & \imprintspan{Hinweise für Autoren finden sich unter derselben Adresse bei Unterpunkt ZChinR / Archiv.} \\
\end{imprint}

\end{document}
