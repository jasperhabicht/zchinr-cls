% !TeX program = xelatex
\documentclass[]{zchinr}

\usepackage{tikz}
%\definecolor{zchinrblue}{cmyk}{1,1,0,.5}
\definecolor{zchinrblue}{cmyk}{1,.5,0,.5}

\zchinrsetup{
  color profile = {
    name = PSO Coated v3,
    info = PSOcoated_v3,
    path = PSOcoated_v3.icc
  },
  issue = {
    year   = 2025,
    number = 1
  }
}

\begin{document}

\zchinrsetup{reset}
\thispagestyle{zchinrempty}

% File: logo_dvjv.tikz 
% Copyright 2024 Jasper Habicht (mail(at)jasperhabicht.de).
% 
% This work may be distributed and/or modified under the
% conditions of the LaTeX Project Public License version 1.3c,
% available at http://www.latex-project.org/lppl/.
% 
% This file is part of the `zchinr' package (The Work in LPPL)
% and all files in that bundle must be distributed together.
% 
% This work has the LPPL maintenance status `maintained'.
% 
\tikzset{
  dcjv logo/.pic={
    \begin{scope}[x=0.1, y=0.1, shift={(-623.3400,-633.3520)}]
      \path[fill=zchinrblue, even odd rule] 
        (1246.6800,633.3520) .. controls
        (1246.6800,289.0900) and (967.5780,9.9883) .. (623.3200,9.9883) .. controls
        (279.0590,9.9883) and (0.0000,289.0900) .. (0.0000,633.3520) .. controls
        (0.0000,977.6090) and (279.0590,1256.6700) .. (623.3200,1256.6700) .. controls
        (967.5780,1256.6700) and (1246.6800,977.6090) .. cycle
      (705.6600,612.1410) .. controls
        (657.8520,602.9610) and (611.0200,593.2700) .. (563.8280,585.4220) .. controls
        (552.4610,583.5120) and (534.6480,583.3520) .. (529.1410,590.0700) .. controls
        (510.4300,612.8400) and (494.1020,638.0780) .. (480.5900,664.2890) .. controls
        (473.3200,678.3910) and (488.9100,682.6090) .. (497.2700,677.0590) .. controls
        (567.1480,630.9690) and (641.4100,665.2190) .. (713.5510,664.8280) .. controls
        (720.8980,664.7890) and (733.1990,676.1210) .. (734.7300,683.8590) .. controls
        (739.7300,708.9690) and (743.4410,734.9100) .. (742.6210,760.3790) .. controls
        (741.8400,785.3400) and (718.8710,784.0510) .. (702.5780,781.7500) .. controls
        (673.9800,777.6480) and (645.6600,770.5390) .. (617.7700,762.6910) .. controls
        (583.4410,753.0390) and (577.0310,755.2190) .. (565.7810,790.3790) .. controls
        (560.9410,805.5000) and (558.5900,821.3980) .. (554.6090,839.2500) .. controls
        (585.5120,830.7700) and (611.2110,821.1210) .. (637.7300,817.2620) .. controls
        (666.0200,813.1210) and (695.1600,813.2300) .. (723.9100,813.7810) .. controls
        (741.0510,814.0900) and (747.5780,825.5000) .. (745.6990,843.0390) .. controls
        (739.1020,904.6020) and (733.3200,966.2380) .. (727.1910,1028.4300) ..
        controls (750.4690,1017.5300) and (776.5590,1007.1800) .. (800.5120,993.1210)
        .. controls (808.2810,988.5390) and (814.9220,973.8980) .. (814.2190,964.4410)
        .. controls (810.8590,920.0310) and (804.7300,875.8090) .. (799.3010,828.7380)
        .. controls (852.9300,847.1410) and (916.3280,836.0120) .. (961.6020,896.1210)
        .. controls (964.0590,879.0900) and (967.1090,870.1480) .. (966.2110,861.5510)
        .. controls (965.0000,849.3280) and (961.0510,837.2890) .. (958.0120,825.2190)
        .. controls (948.1210,785.9690) and (940.9800,782.2890) .. (900.6210,789.6800)
        .. controls (877.0310,793.9690) and (852.9300,795.6480) .. (829.0200,797.7620)
        .. controls (813.9490,799.1290) and (806.0900,791.5510) .. (804.5310,776.1600)
        .. controls (801.4100,744.9100) and (797.5000,713.6990) .. (793.5510,679.6020)
        .. controls (827.6600,684.6800) and (861.8400,688.6600) .. (895.5510,695.1090)
        .. controls (937.3400,703.0780) and (981.7620,703.4300) ..
        (1017.0700,733.5120) .. controls (1020.0400,736.0390) and (1035.9000,730.8090)
        .. (1036.2500,727.9610) .. controls (1040.1600,697.2620) and
        (1042.1900,666.2810) .. (1044.8000,633.9410) .. controls (960.1210,655.2620)
        and (876.3710,635.3010) .. (792.3400,622.9220) .. controls (785.1990,621.8710)
        and (774.2620,614.7890) .. (773.2380,609.0120) .. controls (758.1210,524.3980)
        and (711.3710,460.1090) .. (644.7300,407.4490) .. controls (725.2700,424.2500)
        and (804.2190,440.7300) .. (883.0900,457.1800) .. controls (862.0310,488.8980)
        and (837.6600,525.6210) .. (813.2380,562.2890) .. controls (815.2300,564.9490)
        and (817.2700,567.6090) .. (819.2620,570.2620) .. controls (830.1600,565.5390)
        and (842.3010,562.5310) .. (851.7620,555.7300) .. controls (889.4490,528.5390)
        and (925.0390,498.1910) .. (963.9800,473.0000) .. controls (985.5120,459.1290)
        and (1012.1100,453.1480) .. (1036.4500,443.6210) .. controls
        (1042.3000,441.3200) and (1048.2000,439.0510) .. (1054.1400,436.7500) ..
        controls (1053.3200,432.1800) and (1053.7500,428.8200) .. (1052.3400,427.6480)
        .. controls (1026.0500,405.3400) and (999.8790,382.8400) ..
        (972.7300,361.6290) .. controls (967.8910,357.8400) and (957.7300,357.0200) ..
        (951.6410,359.2890) .. controls (846.1290,398.1910) and (743.9490,373.8980) ..
        (642.0310,343.7810) .. controls (613.3980,335.3010) and (584.6480,327.1800) ..
        (556.6020,317.1410) .. controls (541.8400,311.8710) and (534.6090,317.1800) ..
        (529.8400,329.6020) .. controls (520.3910,354.2890) and (510.8980,378.9690) ..
        (502.2700,403.9410) .. controls (497.4220,417.9220) and (503.7890,427.4100) ..
        (518.2810,427.9610) .. controls (555.9410,429.3710) and (585.5900,447.8790) ..
        (609.1020,474.2500) .. controls (640.9800,509.9490) and (669.5310,548.5780) ..
        (699.1410,586.2380) .. controls (702.6210,590.6480) and (704.3010,596.5900) ..
        (706.2500,602.0200) .. controls (706.9880,604.0510) and (706.0510,606.6720) ..
        cycle
      (261.8790,315.5000) .. controls
        (249.8010,319.5200) and (236.1720,321.5510) .. (225.2700,328.0390) .. controls
        (162.3790,365.5390) and (174.1020,390.3010) .. (215.0390,422.8400) .. controls
        (276.8010,471.8980) and (338.4410,521.0780) .. (400.1990,570.0700) .. controls
        (407.1910,575.6210) and (414.8400,580.3790) .. (422.1480,585.5000) .. controls
        (424.2620,583.8980) and (426.3280,582.2890) .. (428.3980,580.7300) .. controls
        (421.2890,570.4610) and (414.3400,560.1480) .. (407.0700,550.0310) .. controls
        (368.0900,495.6210) and (327.7700,442.1020) .. (290.6600,386.4410) .. controls
        (277.6990,367.0200) and (272.4610,342.4100) .. cycle
      (302.1480,568.3520) .. controls
        (297.3400,598.1910) and (290.6990,627.9220) .. (288.2810,657.9610) .. controls
        (285.6600,690.2190) and (288.0120,723.1480) .. (254.3010,745.3400) .. controls
        (294.6910,742.2620) and (331.6800,736.6720) .. (361.0200,710.9690) .. controls
        (366.4100,706.2810) and (366.9880,690.3400) .. (363.5200,682.3280) .. controls
        (346.7190,643.7380) and (327.7700,606.0390) .. (309.5700,568.0390) .. controls
        (307.0700,568.1480) and (304.6090,568.2300) .. cycle
      (343.9800,917.7620) .. controls
        (378.5900,893.1910) and (409.9610,859.0510) .. (463.8280,875.7300) .. controls
        (451.5200,855.0310) and (440.3520,833.5120) .. (426.5200,813.8980) .. controls
        (415.5510,798.3520) and (401.7190,797.4100) .. (391.5200,815.9690) .. controls
        (373.7500,848.3520) and (356.4800,881.0390) .. (339.0200,913.5780) .. controls
        (340.6600,914.9880) and (342.3010,916.3590) .. cycle;
    \end{scope}
  }
}


\begin{tikzpicture}[
    overlay, remember picture, 
    every node/.append style={
      inner sep=0pt, anchor=north west, align=left, zchinrblue
    }
  ] \sffamily

  \fill[zchinrblue!35] ([shift={(-3mm,-3mm)}]current page.north west) 
    rectangle ([shift={(3mm,-140mm)}]current page.north east);

  \node[white, anchor=north, font=\fontsize{202.5}{202.5}\selectfont\addfontfeature{LetterSpace=-6.75}] 
    at ([shift={(0mm,-22.5mm)}]current page.north) {%
      ZChinR%
    };

  \node[anchor=north, font=\fontsize{72}{65}\selectfont\addfontfeature{LetterSpace=-2.5}] 
    at ([shift={(0mm,-82.5mm)}]current page.north) {%
      Zeitschrift für \\ Chinesisches Recht%
    };

  \node[font=\fontsize{15}{17}\selectfont] 
    at ([shift={(20mm,140mm)}]current page.south west) {%
      \parbox{80mm}{  
        Herausgegeben von der \\
        Deutsch-Chinesischen \\
        Juristenvereinigung e.V. 
        \bigskip\par
        in Verbindung mit dem \\
        Deutsch-Chinesischen Institut \\
        für Rechtswissenschaft 
        \bigskip\par
        und dem Max-Planck-Institut für \\
        ausländisches und internationales \\
        Privatrecht
      }
    };
  
  \pic[scale=0.8] 
    at ([shift={(-35mm,30mm)}]current page.south east) {dcjv logo};

  \node[font=\fontsize{20}{20}\selectfont\bfseries] 
    at ([shift={(20mm,30mm)}]current page.south west) {%
      Heft 1/2025 
    };

  \node[font=\fontsize{11}{8}\selectfont\sffamily\bfseries] 
    at ([shift={(20mm,22.5mm)}]current page.south west) {%
      32. Jahrgang, S. 1--100
    };

  \node[font=\fontsize{13}{15}\selectfont\rmfamily]
    at ([shift={(100mm,140mm)}]current page.south west) {
      \parbox{90mm}{  
        \emph{Franz Kafka,} Die Verwandlung
      }
    };

\end{tikzpicture}


\zchinrsetup{reset}
\thispagestyle{zchinrempty}

\begin{tikzpicture}[
    overlay, remember picture, 
    every node/.append style={
      inner sep=0pt, anchor=north west, align=left, zchinrblue
    }
  ]
  \sffamily
  \fill[zchinrblue!35] ([shift={(-3mm,-3mm)}]current page.north west) 
    rectangle ([shift={(3mm,-140mm)}]current page.north east);

  \node[anchor=north east, align=right, font=\Large\bfseries] 
    at ([shift={(-20mm,-20mm)}]current page.north east) {%
      ISSN: 1613-5768 \\ Online ISSN: 2366-7125
    };

  \node[anchor=north west, align=left, font=\fontsize{72}{65}\selectfont] 
    at ([shift={(20mm,-105mm)}]current page.north west) {%
      ZChinR \char"00B7{} GJCL%
    };

  \node[font=\zchinrsansfamily\Huge]
    at ([shift={(-190mm,140mm)}]current page.south east) {
      Zeitschrift für Chinesisches Recht \\
      German Journal for Chinese Law 
    };

  \node[font=\zchinrsansfamily]
    at ([shift={(-190mm,110mm)}]current page.south east) {
      \parbox{80mm}{  
        \noindent Since 1994 the German--Chinese Jurists' Association and the Sino--German Institute for Legal Studies of the Universities of Göttingen and Nanjing are quarterly publishing the ``Zeitschrift für Chinesisches Recht (German Journal of Chinese Law)'', formerly known as the ``Newsletter of the German-Chinese Jurists' Association''.  \bigskip

        \noindent The journal is focusing on issues of contemporary Chinese law and modern Chinese legal history with a particular emphasis on legal aspects of Chinese economic development and international relations. It seeks to advance practical as well as theoretical analysis of Chinese law.
      }
    };

  \node[font=\zchinrsansfamily]
    at ([shift={(-100mm,110mm)}]current page.south east) {
      \parbox{80mm}{  
        \noindent The journal invites submissions within its scope as set out above to be published in one of its next issues. To guarantee for intellectually stimulating and innovative contributions all submissions will be subject to a review procedure by the editors. Manuscripts (English or German) to be published in the journal's categories articles, short contributions, documentations and book reviews should be submitted in electronic form and should follow the rules of citation and guidelines for the submission of articles, which can be found at www.ZChinR.de. Previous issues of ZChinR can also be found at www.ZChinR.de. \bigskip

        \noindent Please address your manuscripts as well as any inquiries concerning subscription and advertising to:  \bigskip

        \noindent ZChinR, Sino--German Institute for Legal Studies \\
        Nanjing University \\
        Mo. 22, Hankou Lu, 210093 Nanjing, PR China \\
        E-mail: zchinr@dcjv.org \\ 
        Phone\,/\,Fax: +86 25 8663 7892
      }
    };

\end{tikzpicture}

\end{document}